\chapter{Introduction}
Nowadays, Vietnam's population increase dramatically leads to many problems. One of that problems is traffic. In Vietnam, our public transposition is not developing, so that makes our traffic terrible. In the rush hour, on the road, different types of vehicle and vehicle in dense crowd, In this thesis will research on counting multi-type of object then building a vehicle counting application based on density estimation that helps estimate number of vehicle in road. After that, Another application can use that number to notify driver to change the road. \\
In this application applied density-based counting method proposed by Victor Lempitsky \cite{Lempitsky:Zisserman:Destimate}, that is to learn to map the features of objects with their ground truth density. The method is based on supervised learning model as the system must be provided objects has been annotating in images. That information can be made by placing a single dot at each object. Knowing the location of each object, the ground truth density map is generated by accumulating a normalized 2D gaussian kernel centered at each dot annotation. After the map is learned, the number of objects in images can be calculated by directly taking the sum of the density value at each pixel over the image. The advantage of this method is that the system does not need to detect the objects present in the images, which is a difficult task, especially in dense crowd objects. \\
The organization of the thesis is the followings:
\begin{itemize}
	\item Chapter 1 introduces the problem to be researched, the challenges that we may face when we approach this problem and the proposed solution.
	\item Chapter 2 gives an overview of the existing approaches to the counting problem. After that, the chapter presents in details the density estimation framework and the essential background.
	\item Chapter 3 describe application design and implement.
	\item Chapter 4 conclusions And future wWork the thesis.  
\end{itemize}