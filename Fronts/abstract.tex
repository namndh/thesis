\begin{abstract}
\hspace{0.5cm}Nowadays, pedestrians tracking and counting is an extremely active research in the field of computer vision.
It plays a essential role in many applications in video surveillance, self-driving cars. It allows to detect a pedestrian and tracking them
with specific identity during the time of images sequence.\par 
In this thesis, we present techniques that can be used to form a tracking system including trackers, object detection algorithms and 
metrics to evaluate performance of a pedestrian tracking system in common pedestrian tracking dataset.
After studying an online realtime pedestrian tracking method that use advantages of deep learning in pedestrian detecting and extracting appearance 
features - DeepSORT \cite{Wojke2017simple} and related works, we propose two architecture alterations to DeepSORT. Then, we create experiments and comparisons
to examine and evaluate the performance of two proposals and the original method by employing both MOT metrics \cite{Milan2016MOT16AB} and visualizations.
Based on such observation, we point out both advantages and disadvantages of our two proposals comparing to the original method. We also propose some ideas which can be done in the future.\par 



\end{abstract}